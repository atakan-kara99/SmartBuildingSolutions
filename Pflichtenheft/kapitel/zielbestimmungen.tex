\section{Musskriterien}
\begin{itemize}
	\item Es muss ein Accountmanagement vorhanden sein, welches verschiedene Benutzerrollen unterstützt.
	\item Die Software muss dazu in der Lage sein, Vertragsdaten zu übernehmen. Diese werden mittels REST-API von adesso übernommen.
	\item Die Software muss den Baufortschritt sowie Leistungspositionen anzeigen können.
	\item Sowohl Website als auch App müssen eine graphische Nutzeroberfläche anbieten.
	\item Die App muss den Status von Leistungspositionen anzeigen können.
	Diese müssen innerhalb der App auch verändert werden können.
\end{itemize}

\section{Sollkriterien}
\begin{itemize}
	\item Die Website soll folgende Diagramme darstellen können:
		\begin{itemize}
		\item Diagramme über den Baufortschritt eines Projektes
		\item Diagramme über die Zustände der Leistungspositionen aus einem oder mehreren Verträgen
		\end{itemize}
	\item Die Nutzeroberflächen von Desktop-Website und App sollen übersichtlich, gut bedienbar und insgesamt benutzerfreundlich sein.
	\item Die App soll die Möglichkeit anbieten, den Baufortschritt eines Projektes mittels Fotos zu dokumentieren.
	Des Weiteren soll die App offline verwendbar sein, insbesondere soll somit die Fotodokumentation auch offline möglich sein.
\end{itemize}

\section{Kannkriterien}
\begin{itemize}
	\item Die Website kann eine mobile Version der Nutzeroberfläche anzeigen.
	\item Die Fotos zur Dokumentation des Baufortschritts können von anderen Mitarbeitern auf korrekte Durchführung überprüft werden.
	So ist es auch möglich, Kommentare zu Fotos abzugeben und Produktmängel zu melden.
	\item Die Website kann diverse Möglichkeiten anbieten, nach denen Diagramme gefiltert werden können.
	\item Die Website kann Nutzern die Möglichkeit anbieten, aus Leistungspositionen, Verträgen und Projekten eigene Diagramme zu erstellen.
\end{itemize}

\section{Abgrenzungskriterien}
\begin{itemize}
	\item Es wird keine iOS-Version der App geben.
	\item Fotos werden nicht mittels Machine Learning von der Software ausgewertet.
	\item Es wird keine Kompatibilität der Website für veraltete Browserversionen garantiert.
	\item Es wird keine Kompatibilität der App für Androidversionen vor Android 6 garantiert.
\end{itemize}
