\section{Webserver}

Referenzielle Daten werden nicht explizit gelistet, wenn sie ohnehin schon in der Applikation an anderer Stelle gespeicher, bwz. genutzt werden.
Ein Beispiel hierfür sind Projekte und dazugehörige Leistungspositionen. Für die Projektdaten werden hier also keine IDs von zugehörigen Leistungspositionen
als Produkdaten aufgeführt, das diese IDs nur als Referenz genutzt werden und bereits in den Leistungspositionsdaten aufgeführt werden.

\subsection{Übersicht Benutzerrollen}

\begin{figure}[h]
	\centering
	\begin{tabularx}{\textwidth}{| X | X |}
        \hline
		\textbf{Benutzerrolle} & \textbf{ID} \\ \hline \hline
		\textbf{Systemadministrator} & SysAdmin \\ \hline
		\textbf{Organisationsadministrator} & OrgAdmin \\ \hline
        \textbf{Allgemeiner Nutzer der Webapplikation} & User \\ \hline
		\textbf{Andere Nutzerrolle innerhalb einer Organisation} & OrgUser \\ \hline
	\end{tabularx}
	\caption{Benutzerrollen}
	\label{fig:Benutzerrollen}
\end{figure}

\begin{flushleft}
Es werden nur allgemeine Rollen beschrieben. Organisationsadministratoren können eigene Rollen beschreiben und zuweisen. Diese verhalten sich allgemein wie die Rolle \textbf{OrgUser}.
Ein \textbf{OrgUser} kann z.Bsp. nur Daten einer Leistungsposition einsehen, sofern seine Rolle innerhalb seiner Organisation dies zulässt. Ebenso können \textbf{OrgAdmin} beispielsweise nur Verträge und
Positionen ihrer jeweiligen Organisation einsehen.
\end{flushleft}

\subsection{Benutzerdaten}

Die Benutzerdaten umfassen alle Informationen zu einem registrierten Benutzer der Webanwendung. Diese Daten sind nur f\"ur den Benutzer selbst und f\"ur Systemadministratoren einsehbar.

\begin{figure}[h]
	\centering
	\begin{tabularx}{\textwidth}{| X || X | X |}
        \hline
		\textbf{Art der Daten} & \textbf{Beschreibung der Daten} & \textbf{Zugriffsberechtigung} \\ \hline \hline
		\textbf{Benutzerkennung} & Eindeutiger Benutzername und ein Passwort(Hash) & SysAdmin, zugehöriger User (nur Benutzername) \\ \hline
		\textbf{Persönliche Daten} & Name, Nachname und Organisation & SysAdmin, zugehöriger OrgAdmin, zugehöriger User \\ \hline
		\textbf{Benutzerrolle} & Rolle des Benutzers & SysAdmin, zugehöriger OrgAdmin, zugehöriger User \\ \hline
	\end{tabularx}
	\caption{Benutzerdaten}
	\label{fig:Benutzerdaten}
\end{figure}

\newpage

\subsection{Projektdaten}

Die Projektdaten setzen sich aus den Daten der zur Verf\"ugung gestellten REST-API zusammen.

\begin{figure}[h]
	\centering
	\begin{tabularx}{\textwidth}{| X || X | X |}
        \hline
		\textbf{Art der Daten} & \textbf{Beschreibung der Daten} & \textbf{Zugriffsberechtigung} \\ \hline \hline
        \textbf{ID} & ID des Projekts & SysAdmin \\ \hline
		\textbf{Projektname} & Name des Projekts & SysAdmin, zugehöriger OrgAdmin, zugehöriger OrgUser \\ \hline
		\textbf{Projektbeschreibung} & Projektbeschreibungg & SysAdmin, zugehöriger OrgAdmin, zugehöriger OrgUser \\ \hline
		\textbf{Fertigstellungsdatum} & Datum der geplanten Fertigstellung & SysAdmin, zugehöriger OrgAdmin, zugehöriger OrgUser \\ \hline
        \textbf{Adresse} & Adresse/Ort des Projekts & SysAdmin, zugehöriger OrgAdmin, zugehöriger OrgUser \\ \hline
	\end{tabularx}
	\caption{Projektdaten}
	\label{fig:Projektdaten}
\end{figure}

\subsection{Vertragsdaten}

Die Vertragsdaten setzen sich aus den Daten der zur Verf\"ugung gestellten REST-API zusammen. Vertragsdaten werden pro Vertrag gespeichert.

\begin{figure}[h]
	\centering
	\begin{tabularx}{\textwidth}{| X || X | X |}
        \hline
		\textbf{Art der Daten} & \textbf{Beschreibung der Daten} & \textbf{Zugriffsberechtigung} \\ \hline \hline
		\textbf{VertragsID} & ID des Vertrages & SysAdmin \\ \hline
		\textbf{Vertragsname} & Name des Vertrages & SysAdmin, zugehöriger OrgAdmin \\ \hline
	\end{tabularx}
	\caption{Vertragsdaten}
	\label{fig:Vertragsdaten}
\end{figure}

\subsection{Leistungspositionsdaten}

Die Leistungspositionsdaten setzen sich aus den Daten der zur Verf\"ugung gestellten REST-API zusammen und aus dem Status einer jeweiligen Position, welcher über die mobile Applikation geändert werden kann.
Sie werden pro Leistungsposition gespeichert.

\begin{figure}[h]
	\centering
	\begin{tabularx}{\textwidth}{| X || X | X |}
        \hline
		\textbf{Art der Daten} & \textbf{Beschreibung der Daten} & \textbf{Zugriffsberechtigung} \\ \hline \hline
		\textbf{PositionsID} & ID des Vertrages & SysAdmin \\ \hline
		\textbf{Status} & Name des Vertrages & SysAdmin, zugehöriger OrgAdmin, zugehöriger OrgUser \\ \hline
		\textbf{Qualitätsicherungsdaten} & Fotos und dazugehörige Kommentare & SysAdmin, zugehöriger OrgAdmin, zugehöriger OrgUser \\ \hline
	\end{tabularx}
	\caption{Leistungspositionsdaten}
	\label{fig:Leistungspositionsdaten}
\end{figure}