%\begin{tcolorbox}
%In diesem Kapitel werden die folgenden drei Punkte erläutert:
%\begin{enumerate}
%	\item \textit{Anwendungsgebiete:} Was ist der Zweck des Produkts?
%	\item \textit{Zielgruppen:} Für welche Benutzer (oder auch Rollen) ist das Produkt bestimmt?
%	Welche Qualifikationen brauchen die Personen?
%\end{enumerate}
%
%\noindent Die einzelnen Teile des Produkteinsatzes werden üblicherweise als Fließtexte geschrieben.
%\end{tcolorbox}

\textbf{Anwendungsgebiete:}\\
Die Software soll die Verwaltung von Vertragsabläufen und Baufortschritten im Bauwesen vereinfachen.
Diese Prozesse digital durchzuführen, bietet diverse Vorteile.
So kann die Verwaltung etwa effizienter durchgeführt werden, da alles dafür Notwendige anschaulich an einem Ort angezeigt wird - es entsteht kein Papierchaos.\\
Darüber hinaus erhöht die Digitalisierung die Gesamttransparenz der Projekte.
In diesem Zuge können sowohl Auftragnehmer als auch Auftraggeber die Qualität der erbrachten Leistungen einfacher kontrollieren, wodurch Unzufriedenheiten leichter gemeldet und gelöst werden können.\\\\

\noindent \textbf{Zielgruppen:}\\
Die Software wird von Auftragnehmern und deren Mitarbeitern genutzt.
Hierbei hat der Auftragnehmer die Rolle eines Organisationsadministrators und weist seinen Mitarbeitern Rollen innerhalb der Organisation zu.
Außerdem gibt es einen Anwendungsadministrator, der Auftragnehmer wiederum zu Organisationsadministratoren ernennen kann.\\
Sowohl Anwendungs- als auch Organisationsadministratoren sollten leichte technische Kenntnisse besitzen, um die Software gut verwenden zu können.
Die App hingegen erfordert nicht diese Qualifikationen und kann von Mitarbeitern somit einfach bedient werden.
