
\begin{figure}[!h]
	\begin{center}
		\begin{tabularx}{\textwidth}{ p{.05\textwidth} | p{.25\textwidth} | p{.2\textwidth} | X }
			\textbf{Nr.} & \textbf{Anwendungsfall ID} & \textbf{Szenario} & \textbf{Erwartetes Verhalten} \\ \hline
			%%---- Mobile app test cases ----%%
            %%---- Anwendungsfall Push local changes to webserver ----%%
			01 & APP-1 & Der Benutzer loggt sich in die Applikation mit seinen Daten ein. &
			\begin{itemize}
				\item[App:] Der Benutzer erhält eine Übersicht über seine Projekte.
				\item[Web:] Sys-Admin erhält eine Übersicht über die Organisationen.
				\item[Web:] Org-Admin und B.m.b.R erhalten eine Übersicht der Verträge.
			\end{itemize} \\ \hline
			01 & APP-1 & Ein Bild wird einer Leistungsposition vom Nutzer zugeordnet. & Das Bild wird als Review auf den Server geladen und ist für einen Befugten sichtbar. Falls keine Verbindung zum Server besteht, wird das Bild lokal zwischengespeichert. \\ \hline
			02 & APP-1 & Der Nutzer weist einer Leistungsposition einen neuen Status zu. & Der neue Status wird dem Server mitgeteilt und ist für befugte Nutzer sichtbar. Falls keine Verbindung zum Server besteht, wird der neue Status lokal zwischengespeichert. \\ \hline
            03 & APP-1 & Der befugte Nutzer will denn aktuellen Status einer Leistungsposition einsehen & Es wird der aktuelle Status der gewählten Leistungsposition angezeigt. \\ \hline
            04 & APP-1 & Der Nutzer trägt seinen Nutzernamen, sein Passwort und die gültige URL des Webservers ein & Eine neue Verbindung Webserver wird hergestellt. \\ \hline
			05 & ACC-1 & B.m.b.R. betätigt das ''bearbeiten''-Symbol bei den Leistungspositionen. & Daten der Leistungsposition können modifiziert werden und werden dauerhaft gespeichert. \\ \hline
			06 & DIA-1 & B.m.b.R lädt die Vertragsdaten mit den Diagrammen oder die Leistungspositionen. & Es werden automatisch passende Diagramme erzeugt. \\ \hline
			07 & DIA-1 & B.m.b.R nutzt die Filterfunktion bei den Vertragsdaten, Projekten und Leistungspositionen. & Die Ergebnisse werden nach dem entsprechenden Kriterium sortiert. \\ \hline
			%%---- missing test case ----%%
			08 & DIA-2 & missing trigger & triggered action \\ \hline 
		\end{tabularx}	
	\end{center}
	\caption{Beschreibung der Testfälle}
	\label{fig:testfaelle-mobile-app-tabelle}
\end{figure}