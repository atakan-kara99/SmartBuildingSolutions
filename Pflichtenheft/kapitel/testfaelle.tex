\centering
\begin{longtable}[c]{|p{1cm}|p{3cm}|p{4cm}|p{6cm}|}
    \caption{Beschreibung der Testfälle}
    \label{fig:testfälle}
    \endlastfoot
    \hline \multicolumn{4}{|r|}{{Weitergeführt auf der folgenden Seite}}                                                                                                                                                                                                                                                                        \\ \hline
    \endfoot
    \hline
    \textbf{Nr.} & \textbf{Anwendungsfall ID} & \textbf{Szenario}                                                                                                  & \textbf{Erwartetes Verhalten}                                                                                                                                              \\ \hline
    \endhead
    \hline
    01           & APP-1                      & Der User loggt sich in die Applikation mit seinen Daten ein.                                                       &
    \begin{itemize}
        \item[App:] Der AppUser erhält eine Übersicht über seine Projekte.
        \item[Web:] Der SysAdmin erhält eine Übersicht über die Organisationen.
        \item[Web:] Der OrgAdmin und B.m.b.R. erhalten eine Übersicht der Verträge.
    \end{itemize}                                                                                                                                                                                                                                                                                                                   \\ \hline
    %%---- APP-1: Ablauf ----%%
    02           & APP-1                      & Der Nutzer weist einer Leistungsposition einen neuen Status zu.                                                    & Der neue Status wird dem Server mitgeteilt und ist für befugte Nutzer sichtbar. Falls keine Verbindung zum Server besteht, wird der neue Status lokal zwischengespeichert. \\ \hline
    %%---- APP-1: Alternative 1 ----%%
    03           & APP-1                      & Ein Bild wird einer Leistungsposition vom Nutzer zugeordnet.                                                       & Das Bild wird als Review auf den Server geladen und ist für einen Befugten sichtbar. Falls keine Verbindung zum Server besteht, wird das Bild lokal zwischengespeichert.   \\ \hline
    04           & APP-1                      & Der befugte Nutzer will den aktuellen Status einer Leistungsposition einsehen.                                     & Es wird der aktuelle Status der gewählten Leistungsposition angezeigt.                                                                                                     \\ \hline
    05           & APP-1                      & Der AppUser trägt seinen Nutzernamen, sein Passwort und die gültige URL des WebServers ein.                        & Eine neue Verbindung zu einem WebServer wird hergestellt.                                                                                                                  \\ \hline
    06           & ACC-1                      & Der OrgAdmin erstellt eine neue Rolle mit gewählten Rechten.                                                       & Die neue Rolle ist erstellt und hat die gewählten Rechte für bestimmte Bereiche auf dem WebServer.                                                                         \\ \hline
    07           & ACC-1                      & Der OrgAdmin fügt neue Nutzer hinzu oder löscht bestehende.                                                        & Neue Nutzer werden dauerhaft Organisationen mit einer Rolle zugewiesen. Zu löschende Nutzer werden dauerhaft aus der Organisation gelöscht.                                \\ \hline
    08           & ACC-1                      & Der SysAdmin erstellt eine neue Organisation mit zugehörigen OrgAdmins.                                            & Nach Erstellung der Organisationen können die OrgAdmins beliebig Nutzer hinzufügen, bearbeiten oder löschen.                                                               \\ \hline
    %%---- ACC-1: Ablauf ----%%
    09           & ACC-1                      & B.m.b.R. bearbeitet Leistungspositionen.                                                                           & Daten der Leistungsposition können erfolgreich modifiziert und dauerhaft gespeichert werden.                                                                               \\ \hline
    10           & ACC-1                      & B.m.b.R. hat Einsicht in die Vertragsdaten.                                                                        & Aus dieser Übersicht können Diagramme erstellt und Vertragsdaten geändert werden.                                                                                          \\ \hline
    %%---- DIA-1: Ablauf ----%%
    11           & DIA-1                      & B.m.b.R. wählt ein alle Leistungspositionen umfassendes Diagramm eines oder mehrerer Verträge zur Darstellung.     & Das passende Diagramm wird angezeigt.                                                                                                                                      \\ \hline
    12           & DIA-1                      & B.m.b.R. lädt die Vertragsdaten mit den Diagrammen oder die Leistungspositionen.                                   & Es werden automatisch passende Diagramme erzeugt.                                                                                                                          \\ \hline
    %%---- DIA-1: Alternative 1 ----%%
    13           & DIA-1                      & B.m.b.R. nutzt die Filterfunktion bei den Vertragsdaten, Projekten und Leistungspositionen.                        & Die Ergebnisse werden nach dem entsprechenden Kriterium gefiltert angezeigt.                                                                                               \\ \hline
    %%---- DIA-1: Alternative 2 ----%%
    14           & DIA-1                      & B.m.b.R. wählt ein Diagramm zum Baufortschritt eines bestimmten Projektes.                                         & Das gewählte Diagramm wird angezeigt.                                                                                                                                      \\ \hline
    %%---- DIA-2: Ablauf ----%%
    15           & DIA-2                      & B.m.b.R. wählt aus einer Liste Leistungspositionen, Verträge und Projekte aus, um ein Diagramm anzeigen zu lassen. & Es wird ein zur Auswahl passendes Diagramm angezeigt.                                                                                                                      \\ \hline
\end{longtable}