
%%%%---- test cases for the mobile/web application ----%%%%
\section{Testfälle für Mobile-/Webapp}
\begin{figure}[!h]
	\begin{center}
		\begin{tabularx}{\textwidth}{ p{.05\textwidth} | p{.25\textwidth} | p{.2\textwidth} | X }
			\textbf{Nr.} & \textbf{Anwendungsfall ID} & \textbf{Szenario} & \textbf{Erwartetes Verhalten} \\ \hline
			01 & bapp-1 & Der Benutzer loggt sich in die Applikation mit seinen Daten ein. & 
			\begin{itemize}
				\item[App:] Der Benutzer erhält eine Übersicht über seine Projekte.
				\item[Web:] Sys-Admin erhält eine Übersicht über die Organisationen.
				\item[Web:] Org-Admin und B.m.b.R erhalten eine Übersicht der Verträge.
			\end{itemize} \\ \hline
			02 & bapp-2 & B.m.b.R. wählt ein Projekt X aus. & Es erscheint eine Übersicht der Leistungspositionen des Projektes X. \\ \hline
			03 & bapp-3 & B.m.b.R. klickt auf eine Leisungsposition Y. & Detailierte Informationen über die Leistungsposition Y werden bereitgestellt. \\ \hline
			04 & bapp-4 & B.m.b.R. betätigt das ''bearbeiten''-Symbol. & Daten der Leistungsposition können modifiziert werden. \\ \hline
			05 & bapp-5 & Der Benutzer drückt auf den Link zum Ausloggen. & Der Benutzer wird erfolgreich ausgeloggt. \\ \hline
		\end{tabularx}	
	\end{center}
	\caption{Beschreibung der Testfälle}
	\label{fig:testfaelle-beide-anwendungen-tabelle}
\end{figure}

%%%%---- test cases for the web application ----%%%%
\newpage
\section{Testfälle für Webapp}
\begin{figure}[!h]
	\begin{center}
		\begin{tabularx}{\textwidth}{ p{.05\textwidth} | p{.25\textwidth} | p{.2\textwidth} | X }
			\textbf{Nr.} & \textbf{Anwendungsfall ID} & \textbf{Szenario} & \textbf{Erwartetes Verhalten} \\ \hline
			%%---- Web app test cases ----%%
			01 & wapp-1 & Der Sys-Admin klickt auf Organisation hinzufügen. & Es wird eine neue Organisation erstellt, in der Admins hinzugefügt, bearbeitet und gelöscht werden können. \\ \hline
			02 & wapp-2 & Der Sys-Admin klickt auf Admin hinzufügen. & Es erscheint ein Fenster, in der ein Admin zu der ausgewählten Organisation hinzugefügt werden kann. \\ \hline
			03 & wapp-3 & B.m.b.R. sucht nach einem Vertrag/einem Projekt/einer Leistungsposition. & Es erscheint ein Vertrag/ein Projekt/eine Leistungsposition, der/die zur Suche passt \\ \hline
			04 & wapp-4 & B.m.b.R. filtern die Verträge/die Projekte/ die Leistungspositionen. & Es werden nur Verträge/Projekte/Leistungspositionen angezeigt, die das Kriterium erfüllen \\ \hline
			05 & wapp-5 & B.m.b.R. drücken auf ''Diagramme erstellen''. & Zu jedem Vertrag wird ein passendes Diagramm erstellt \\ \hline
			06 & wapp-6 & B.m.b.R. drücken auf Vertragsdaten. & Zu jedem Vertrag werden die Daten im Textformat angezeigt. \\ \hline 
			07& wapp-7 & Org-Admin setzt einen Haken beim Nutzer und drückt anschließend auf ausgewählte(n) Nutzer löschen. & Der ausgewählte Nutzer wird dauerhaft gelöscht \\ \hline
			08 & wapp-8 & Org-Admin drückt auf Benutzer hinzufügen. & Ein Formular erscheint, indem ein weiterer Benutzer hinzugefügt werden. \\ \hline
			09 & wapp-9 & Org-Admin drückt auf ''Rollen verwalten''. & Er erhält eine Übersicht über alle Rollen, mit deren Rechten und kann diese Rollen ebenfalls bearbeiten. \\ \hline
			10 & wapp-10 & Org-Admin drückt auf ''Rolle erstellen''. & Ein Formular erscheint, bei dem eine Rolle erstellt werden kann und dessen Rechte zugewiesen werden können. \\ \hline
		\end{tabularx}	
	\end{center}
	\caption{Beschreibung der Testfälle}
	\label{fig:testfaelle-web-app-tabelle}
\end{figure}

%%%%---- test cases for the mobile application ----%%%%
\newpage

\section{Testfälle für Mobileapp}

\begin{figure}[!h]
	\begin{center}
		\begin{tabularx}{\textwidth}{ p{.05\textwidth} | p{.25\textwidth} | p{.2\textwidth} | X }
			\textbf{Nr.} & \textbf{Anwendungsfall ID} & \textbf{Szenario} & \textbf{Erwartetes Verhalten} \\ \hline
			%%---- Mobile app test cases ----%%
			01 & mapp-1 & Ein Bild wird einer Leistungsposition vom Nutzer zugeordnet. & Das Bild wird als Review auf den Server geladen und ist für einen Befugten sichtbar. Falls keine Verbindung zum Server besteht, wird das Bild lokal zwischengespeichert. \\ \hline
			02 & mapp-2 & Der Nutzer weist einer Leistungsposition einen neuen Status zu. & Der neue Status wird dem Server mitgeteilt und ist für befugte Nutzer sichtbar. Falls keine Verbindung zum Server besteht, wird der neue Status lokal zwischengespeichert. \\ \hline
            03 & mapp-3 & Der befugte Nutzer will denn aktuellen Status einer Leistungsposition einsehen & Es wird der aktuelle Status der gewählten Leistungsposition angezeigt. \\ \hline
            04 & mapp-4 & Der Nutzer trägt seinen Nutzernamen, sein Passwort und die gültige URL des Webservers ein & Eine neue Verbindung Webserver wird hergestellt. \\ \hline
		\end{tabularx}	
	\end{center}
	\caption{Beschreibung der Testfälle}
	\label{fig:testfaelle-mobile-app-tabelle}
\end{figure}