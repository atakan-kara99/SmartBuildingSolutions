\begin{tcolorbox}
In diesem Abschnitt werden Testfälle für die Anwendungsfälle der Produktfunktionen definiert.
Diese sollen später ebenfalls als \textbf{reale Tests} implementiert werden.

\autoref{fig:testfaelle-tabelle} stellt eine exemplarische Tabelle für die Beschreibung der zu testenden Anwendungsfälle dar. 
Stil und Formatierung sind variabel.
\end{tcolorbox}

\begin{figure}[!h]
	\begin{center}
		\begin{tabularx}{\textwidth}{ p{.05\textwidth} | p{.25\textwidth} | p{.2\textwidth} | X }
			\textbf{Nr.} & \textbf{Anwendungsfall ID} & \textbf{Szenario} & \textbf{Erwartetes Verhalten} \\ \hline
			1 & mapp-1 & Der Benutzer loggt sich in die Applikation mit Username, Passwort und Server-URL ein. & Bei erfolgreichem Login erhält der Benutzer eine Übersicht über alle mit ihm verbunden Projekte. Bei Misserfolg wird eine Fehlermeldung beim Login-Screen angezeigt. \\ \hline
			2 & mapp-2 & Der Benutzer klickt auf ein Projekt X. & Der Nutzer erhält die Übersicht über das Projekt X. \\ \hline
			3 & mapp-3 & Der Benutzer klickt auf eine Leistungsposition Y von Projekt X. & Der Benutzer erhält eine Übersicht über die spezielle Leistungsposition. \\ \hline
			4 & mapp-4 & Bei der Leistungsposition wird ein Bild hinzugefügt. & Das Bild wird bei einer vorhanden Internet-Verbindung zum Server gesendet. Falls noch keine Verbindung vorhanden ist: das Bild wird bis zur erneuten Verbindung zwischengespeichert. \\ \hline
			5 & mapp-5 & Benutzer klickt auf das WLAN-Symbol & Es erscheint ein Fenster, bei dem ein neuer Server eingetragen werden kann. \\ \hline
			6 & mapp-6 & Der Nutzer trägt einen neuen Server ein. & Der Verbindung soll zum neuen Server während der Laufzeit der App hergestellt werden. \\ \hline
			7 & mapp-7 & Klicken auf das Logout-Symbol. & Der Benutzer loggt sich erfolgreich aus der App aus. \\ \hline
		\end{tabularx}	
	\end{center}
	\caption{Beschreibung der Testfälle}
	\label{fig:testfaelle-tabelle}
\end{figure}
