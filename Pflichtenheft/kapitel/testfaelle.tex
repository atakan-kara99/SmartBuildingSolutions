\begin{tcolorbox}
In diesem Abschnitt werden Testfälle für die Anwendungsfälle der Produktfunktionen definiert.
Diese sollen später ebenfalls als \textbf{reale Tests} implementiert werden.

\autoref{fig:testfaelle-tabelle} stellt eine exemplarische Tabelle für die Beschreibung der zu testenden Anwendungsfälle dar. 
Stil und Formatierung sind variabel.
\end{tcolorbox}

\begin{figure}[!h]
	\begin{center}
		\begin{tabularx}{\textwidth}{ p{.05\textwidth} | p{.25\textwidth} | p{.2\textwidth} | X }
			\textbf{Nr.} & \textbf{Anwendungsfall ID} & \textbf{Szenario} & \textbf{Erwartetes Verhalten} \\ \hline
			%%---- General test cases ----%%
			1 & bapp-1 & Der Benutzer loggt sich in die Applikation mit seinen Daten ein. & App:\r -> Der Benutzer erhält eine Überischt über seine Projekte.\r Web: \r -> Sys-Admin erhält eine Übersicht über die Organisationen. \r -> Orga-Admin und B.m.b.R erhalten eine Übersicht der Verträge. \\ \hline
			2 & bapp-2 & B.m.b.R wählt ein Projekt X aus. & Es erscheint eine Übersicht der Leistungspositionen des Projektes X. \\ \hline
			3 & bapp-3 & B.m.b.R klickt auf eine Leisungsposition Y. & Detailierte Informationen über die Leistungsposition Y werden bereitgestellt. \\ \hline
			4 & bapp-4 & B.m-b.R betätigt das "bearbeiten"-Symbol. & Daten der Leistungsposition können modifiziert werden. \\ \hline
			5 & bapp-5 & Der Benutzer drückt auf den Link zum Ausloggen. & Der Benutzer wird erfolgreich ausgeloggt. \\ \hline
		\end{tabularx}	
	\end{center}
	\caption{Beschreibung der Testfälle}
	\label{fig:testfaelle-beide-anwendungen-tabelle}
\end{figure}

\begin{figure}[!h]
	\begin{center}
		\begin{tabularx}{\textwidth}{ p{.05\textwidth} | p{.25\textwidth} | p{.2\textwidth} | X }
			\textbf{Nr.} & \textbf{Anwendungsfall ID} & \textbf{Szenario} & \textbf{Erwartetes Verhalten} \\ \hline
			%%---- Mobile app test cases ----%%
			4 & mapp-4 & Bei der Leistungsposition wird ein Bild hinzugefügt. & Das Bild wird bei einer vorhanden Internet-Verbindung zum Server gesendet. Falls noch keine Verbindung vorhanden ist: das Bild wird bis zur erneuten Verbindung zwischengespeichert. \\ \hline
			5 & mapp-5 & Der Nutzer trägt einen neuen Server ein. & Eine neue Verbindung zu einem anderen Server wird hergestellt. \\ \hline
		\end{tabularx}	
	\end{center}
	\caption{Beschreibung der Testfälle}
	\label{fig:testfaelle-mobile-app-tabelle}
\end{figure}

\begin{figure}[!h]
	\begin{center}
		\begin{tabularx}{\textwidth}{ p{.05\textwidth} | p{.25\textwidth} | p{.2\textwidth} | X }
			\textbf{Nr.} & \textbf{Anwendungsfall ID} & \textbf{Szenario} & \textbf{Erwartetes Verhalten} \\ \hline
			%%---- Web app test cases ----%%
			8 & wapp-1 & Der Benutzer loggt sich in die Webanwendung ein & Der System-Admin erhält eine Übersicht über die Organisationen. \r Der Orga-Admin und alle weiteren Befugten erhalten eine Übersicht über die einzusehenden Verträge. \\ \hline
			9 & wapp-2 & Der Systemadmin klickt auf Organisation hinzufügen. & Es wird eine neue Organisation erstellt, in der Admins hinzugefügt, bearbeitet und gelöscht werden können. \\ \hline
			10 & wapp-3 & Der Systemadmin klickt auf Admin hinzufügen. & Es erscheint ein Fenster, in der ein Admin zu der ausgewählten Organisation hinzugefügt werden kann. \\ \hline
			11 & wapp-4 & Benutzer mit bestimmten Rechten sucht nach einem Vertrag & Es erscheint ein Vertrag, der zur Suche passt \\ \hline
			12 & wapp-5 & Benutzer mit bestimmten Rechten filtern die Verträge & Es werden nur Verträge angezeigt, die das Kriterium erfüllen \\ \hline
			13 & wapp-6 & Benutzer mit bestimmten Rechten drücken auf "Diagramme erstellen" & Zu jedem Vertrag wird ein passendes Diagramm erstellt \\ \hline
			14 & wapp-7 & Benutzer mit bestimmten Rechten sucht nach einem Projekt & Es erscheint ein Projekt, welches zur Suche passt! \\ \hline
			15 & wapp-8 & Benutzer mit bestimmten Rechten filtert alle Projekte & Es werden nur Projekte angezeigt, die das Kriterium erfüllen \\ \hline
			16 & wapp-9 & Benutzer mit bestimmten Rechten klickt auf ein Projekt & Es werden die Leistungspositionen zu diesem Projekt angezeigt \\ \hline
			17 & wapp-10 & Benutzer mit bestimmten Rechten sucht nach einer Leistungsposition & Es erscheint eine Leistungsposition, die zur Suche passt. \\ \hline
			18 & wapp-11 & Benutzer mit bestimmten Rechten filtert alle Leistungspositionen & Es werden nur die Leistungspositionen angezeigt, die das Kriterium erfüllen \\ \hline
			19 & wapp-12 & Benutzer mit bestimmten Rechten drückt auf eine Leistungsposition & Es werden alle Informationen zu dieser Leistungspositionen angezeigt \\ \hline
			20 & wapp-14 & Benutzer mit bestimmten Rechten drückt auf Leistungsposition bearbeiten & Leistungsposition kann bearbeitet und reviewt werden \\ \hline
			21 & wapp-15 & Benutzer mit bestimmten Rechten setzt einen Haken beim Nutzer und drückt anschließend auf ausgewählte(n) Nutzer löschen. & Der ausgewählte Nutzer wird dauerhaft gelöscht \\ \hline
			22 & wapp-16 & Benutzer mit bestimmten Rechten drückt auf Benutzer hinzufügen. & Ein Formular erscheint, indem ein weiterer Benutzer hinzugefügt werden. \\ \hline
			23 & wapp-17 & Benutzer mit bestimmten Rechten drückt auf "Rollen verwalten". & Er erhält eine Übersicht über alle Rollen, mit deren Rechten und kann diese Rollen ebenfalls bearbeiten. \\ \hline
			24 & wapp-18 & Benutzer mit bestimmten Rechten drückt auf "Rolle erstellen". & Ein Formular erscheint, bei dem eine Rolle erstellt werden kann und dessen Rechte zugewiesen werden können. \\ \hline
			25 && wapp-19 & (System-)Admins und Benutzer drücken auf Logout & Der (System-)Admin/Benutzer wird erfolgreich ausgeloggt.
		\end{tabularx}	
	\end{center}
	\caption{Beschreibung der Testfälle}
	\label{fig:testfaelle-web-app-tabelle}
\end{figure}
