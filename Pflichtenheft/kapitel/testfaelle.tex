
%%%%---- test cases for the mobile/web application ----%%%%
\section{Testfälle für Mobile-/Webapp}
\begin{figure}[!h]
	\begin{center}
		\begin{tabularx}{\textwidth}{ p{.05\textwidth} | p{.25\textwidth} | p{.2\textwidth} | X }
			\textbf{Nr.} & \textbf{Anwendungsfall ID} & \textbf{Szenario} & \textbf{Erwartetes Verhalten} \\ \hline
			01 & LOG-1 & Der Benutzer loggt sich in die Applikation mit seinen Daten ein. &
			\begin{itemize}
				\item[App:] Der Benutzer erhält eine Übersicht über seine Projekte.
				\item[Web:] Sys-Admin erhält eine Übersicht über die Organisationen.
				\item[Web:] Org-Admin und B.m.b.R erhalten eine Übersicht der Verträge.
			\end{itemize} \\ \hline
			02 & PRO-1 & B.m.b.R wählt ein Projekt X aus. & Es erscheint eine Übersicht der Leistungspositionen des Projektes X. \\ \hline
			03 & LEI-1 & B.m.b.R klickt auf eine Leisungsposition Y. & Detailierte Informationen über die Leistungsposition Y werden bereitgestellt. \\ \hline
			%%---- Anwendungsfall Change status of position ----%%
			04 & ACC-1 & B.m-b.R betätigt das "bearbeiten"-Symbol. & Daten der Leistungsposition können modifiziert werden. \\ \hline
			05 & LOG-2 & Der Benutzer drückt auf den Link zum Ausloggen. & Der Benutzer wird erfolgreich ausgeloggt. \\ \hline
		\end{tabularx}	
	\end{center}
	\caption{Beschreibung der Testfälle}
	\label{fig:testfaelle-beide-anwendungen-tabelle}
\end{figure}

%%%%---- test cases for the web application ----%%%%
\section{Testfälle für Webapp}
\begin{figure}[!h]
	\begin{center}
		\begin{tabularx}{\textwidth}{ p{.05\textwidth} | p{.25\textwidth} | p{.2\textwidth} | X }
			\textbf{Nr.} & \textbf{Anwendungsfall ID} & \textbf{Szenario} & \textbf{Erwartetes Verhalten} \\ \hline
			%%---- Web app test cases ----%%
			01 & ORG-1 & Der Sys-Admin klickt auf Organisation hinzufügen. & Es wird eine neue Organisation erstellt, in der Admins hinzugefügt, bearbeitet und gelöscht werden können. \\ \hline
			02 & ORG-2 & Der Sys-Admin klickt auf Admin hinzufügen. & Es erscheint ein Fenster, in der ein Admin zu der ausgewählten Organisation hinzugefügt werden kann. \\ \hline
			%%---- Anwendungsfall Diagrammverwaltung ----%%
			03 & DIA-1 & B.m.b.R sucht nach einem Vertrag/einem Projekt/einer Leistungsposition. & Es erscheint ein Vertrag/ein Projekt/eine Leistungsposition, der/die zur Suche passt. \\ \hline
			%%---- Anwendungsfall Diagrammverwaltung ----%%
			04 & DIA-1 & B.m.b.R filtern die Verträge/die Projekte/ die Leistungspositionen. & Es werden nur Verträge/Projekte/Leistungspositionen angezeigt, die das Kriterium erfüllen. \\ \hline
			%%---- Anwendungsfall Diagrammverwaltung ----%%
			05 & DIA-1 & B.m.b.R drücken auf "Diagramme erstellen". & Zu jedem Vertrag wird ein passendes Diagramm erstellt. \\ \hline
			06 & VER-1 & B.m.b.R drücken auf Vertragsdaten. & Zu jedem Vertrag werden die Daten im Textformat angezeigt. \\ \hline 
			07 & ADM-1 & Org-Admin setzt einen Haken beim Nutzer und drückt anschließend auf ausgewählte(n) Nutzer löschen. & Der ausgewählte Nutzer wird dauerhaft gelöscht. \\ \hline
			08 & ADM-2 & Org-Admin drückt auf Benutzer hinzufügen. & Ein Formular erscheint, indem ein weiterer Benutzer hinzugefügt werden. \\ \hline
			09 & ADM-3 & Org-Admin drückt auf "Rollen verwalten". & Er erhält eine Übersicht über alle Rollen, mit deren Rechten und kann diese Rollen ebenfalls bearbeiten. \\ \hline
			10 & ADM-4 & Org-Admin drückt auf "Rolle erstellen". & Ein Formular erscheint, bei dem eine Rolle erstellt werden kann und dessen Rechte zugewiesen werden können. \\ \hline
		\end{tabularx}	
	\end{center}
	\caption{Beschreibung der Testfälle}
	\label{fig:testfaelle-web-app-tabelle}
\end{figure}

%%%%---- test cases for the mobile application ----%%%%
\newpage
\section{Testfälle für Mobileapp}

\begin{figure}[!h]
	\begin{center}
		\begin{tabularx}{\textwidth}{ p{.05\textwidth} | p{.25\textwidth} | p{.2\textwidth} | X }
			\textbf{Nr.} & \textbf{Anwendungsfall ID} & \textbf{Szenario} & \textbf{Erwartetes Verhalten} \\ \hline
			%%---- Mobile app test cases ----%%
			%%---- Anwendungsfall Push local changes to webserver ----%%
			01 & APP-1 & Ein Bild wird einer Leistungsposition zugeordnet. & Das Bild wird als Review auf den Server geladen und ist für einen Befugten sichtbar. \\ \hline
			02 & SER-1 & Der Nutzer trägt einen neuen Server ein. & Eine neue Verbindung zu einem anderen Server wird hergestellt. \\ \hline
		\end{tabularx}	
	\end{center}
	\caption{Beschreibung der Testfälle}
	\label{fig:testfaelle-mobile-app-tabelle}
\end{figure}