%\begin{tcolorbox}
%Teilt eure Klassendiagramme bitte auf und baut \textbf{kein} einzelnes riesiges Diagramm.
%Getter und Setter Methoden müssen hier nicht modelliert werden.
%Sie sollten aber der klassischen Namenskonvention folgen, um die Nutzung in Sequenzdiagrammen zu ermöglichen.
%\\\\
%Auf jedes Diagramm folgt eine Tabelle, in der die Aufgabe \textbf{jeder} Klasse beschrieben wird.
%\end{tcolorbox}

\section{Backend}

\subsection{Backend Datenmodell}

\begin{figure}[H]
	\centering
	\includegraphics[width=\linewidth]{img/diagrams/class-diagram-backend.pdf}
	\caption{Klassendiagramm - Backend}
	\label{fig:klassendiagramm-backend}
\end{figure}

\clearpage

\noindent
Dieses Klassendiagramm enthält die Entitäten der Datenbank, jede Klasse entspricht einer Entität. \\
Die Klasse Role besitzt zu den Klassen BillingItem, Contract, User, Organisation und Project min. 2 Rollen. \\
Dies ist damit zu begründen, dass der SysAdmin Teil jeder Organisation ist und somit alle Projekte, User, Verträge und Leistungspositionen einsehen kann.
Zusätzlich existiert zu jeder Organisation min. ein OrgAdmin, der ebenfalls auf alle Verträge, Projekte und Leistungspositionen zugreifen kann.
 
\begin{longtable}[h]{p{5.3cm} p{8.7cm}}
	\caption{Klassenbeschreibung - Backend}
	\label{table:klassenbeschreibung-backend}
    \endlastfoot
	\multicolumn{2}{r}{{Weitergeführt auf der folgenden Seite}} \\
	\endfoot
	\endhead
	\rowcolor[HTML]{C0C0C0} 
	\textbf{Klassenname} & \textbf{Aufgabe} \\
    
	Address & Die Adresse des Projektes. \\
	
	\rowcolor[HTML]{E7E7E7} 
	BillingItem & Die Leistungspositionen eines Vertrages. \\
	
	BillingUnit & Eine Gruppierung von Leistungspositionen. \\
	
	\rowcolor[HTML]{E7E7E7} 
	Contract & Ein Vertrag, welcher zwischen zwei Parteien geschlossen wird. \\
	
	Organisation & Eine Gruppierung von Usern. \\
	
	\rowcolor[HTML]{E7E7E7} 
	Project & Ein Projekt min. einer Organisation, welches mehrere Verträge enhalten kann. \\
	
	Role & Nutzerrollen mit verschiedenen Rechten. Verwaltbar vom OrgAdmin. \\

	\rowcolor[HTML]{E7E7E7}
	User & Mitarbeiter min. einer Organisation.
\end{longtable}

\subsection{Backend Datenverarbeitung}

\begin{figure}[H]
	\centering
	\includegraphics[width=\linewidth]{img/diagrams/Backend.pdf}
	\caption{Klassendiagramm - Backend - Datenverarbeitung}
	\label{fig:klassendiagramm-backend-data}
\end{figure}

\clearpage

\begin{center}
\begin{longtable}[h]{p{5cm} p{9cm}}
	\caption{Klassenbeschreibung - Backend - Datenverarbeitung}
	\label{table:klassenbeschreibung-backend-data}
	\endlastfoot
	\multicolumn{2}{r}{{Weitergeführt auf der folgenden Seite}} \\
	\endfoot
	\endhead
	\rowcolor[HTML]{C0C0C0} 
	\textbf{Klassenname} & \textbf{Aufgabe} \\
    CrudRepository & Teil des Spring Frameworks. Wird hier genutzt, um die Datenbank in From von erbenden Repositories darzustellen \\
	\rowcolor[HTML]{E7E7E7} 
	*Repository & Verwaltet die jeweilige Datenmodellklasse. Bietet spezielle Zugriffsfunktionen für eine einfachere Nutzung \\
	BackendAccessProvider & Bietet die eigentliche Funktionalität des Backends innerhalb der Server-Applikation an. Über diese Klasse wird der gesicherte Zugriff auf die gespeicherten Daten sichergestellt und die 
    einzelnen Daten-Repositories werden vor dem Nutzer verborgen. Alle Methoden verlangen einen Nutzernamen als Parameter, um festzustellen welche Daten zurückgegeben werden dürfen. Dafür werden intern die Rollen verwendet.
    So soll ein Nutzer z.B. nur Zugriff auf Verträge bekommen für welche er über eine entsprechende Rolle verfügt. Ansonsten werden leere Listen und auch Fehlercodes zurückgegeben, welche dann z.B. vom Frontend
    entsprechend verarbeitet werden können. Die Controller-Klassen des Frontends haben folglich Zugriff auf den BackendAccessProvider. \\
	\rowcolor[HTML]{E7E7E7} 
	DBSynchronisationService & Dienst, welcher die Aufgabe hat nach einem Zeitintervall eine Synchronisation zwischen der Datenbank von adesso und der lokalen zu veranlassen. Die Eigentliche Synchronisation wird
    durch die Klasse RESTDataRetriever durchgeführt. \\
    RESTDataRetriever & Hat zur Aufgabe die Daten über die REST-API von adesso abzufragen und folgend die Daten zu deserialisieren. Die so erhaltenen Klassen werden abschließend in die Datenbank über den BackendAccessProvider eingepflegt. \\
	\rowcolor[HTML]{E7E7E7} 
    JSONDeserialiser & Konvertiert JSON-Strings in die entsprechenden Modellklassen. Dies findet Verwendung, wenn die Daten über die REST-API von adesso abgefragt werden. \\
\end{longtable}
\end{center}

\clearpage

\section{Web}

\begin{figure}[h]
	\centering
	\includegraphics[width=\linewidth]{img/diagrams/Frontend Classes.pdf}
	\caption{Klassendiagramm - Frontend Web}
	\label{fig:klassendiagramm-web}
\end{figure}

\noindent
Klassen, deren Name mit ''Controller'' aufhört, verarbeiten HTTP Requests zu bestimmten Pfaden.
Die zu verarbeitenden Pfade sind pro Gebiet in einem jeweiligen Controller gruppiert.
In der folgenden Tabelle werden für Controller unter ''Aufgabe'' die zu verarbeitenden Pfade sowie die Bedeutung der dazugehörigen Seite aufgeführt.
Die Identifikationsnummern oID (Organisation), diaID (Diagramm), pID (Projekt), cID (Vertrag) und bID (Leistungsposition) sind in manche Pfade direkt integriert.\\

\begin{longtable}[h]{p{5.3cm} p{8.7cm}}
	\caption{Klassenbeschreibung - Frontend Web}
	\label{table:klassenbeschreibung-web}
	\endlastfoot
	\multicolumn{2}{r}{{Weitergeführt auf der folgenden Seite}} \\
	\endfoot
	\endhead
	\rowcolor[HTML]{C0C0C0} 
	\textbf{Klassenname} & \textbf{Aufgabe} \\
    
	DispatcherServlet & Teil des Spring Frameworks, leitet die HTTP Requests an den jeweils zuständigen Controller weiter \\
	
	\rowcolor[HTML]{E7E7E7} 
	LoginController & /login $\rightarrow$ Login-Seite \\
	
	OrganisationManagementController & /organisation\_overview $\rightarrow$ Management von Organisationen und deren OrgAdmins, nur der SysAdmin hat hierauf Zugriff \\
	
	\rowcolor[HTML]{E7E7E7} 
	UserManagementController & /organisation/\{oID\}/user\_management $\rightarrow$ Management der WebUser einer Organisation \newline\newline
	/organisation/\{oID\}/user\_management/user\_new $\rightarrow$ Hinzufügen eines WebUsers zu einer Organisation \newline\newline
	/organisation/\{oID\}/user\_management/user/\{uID\}/user\_edit $\rightarrow$ Bearbeiten eines WebUsers einer Organisation \\
	
	RoleManagementController & /organisation/\{oID\}/role\_management $\rightarrow$ Management der Rollen einer Organisation \newline\newline
	/organisation/\{oID\}/role\_management/role\_new $\rightarrow$ Hinzufügen einer Rolle zu einer Organisation \newline\newline
	/organisation/\{oID\}/role\_management/role/\{rID\}/role\_edit $\rightarrow$ Bearbeiten einer Rolle einer Organisation \\
	
	\rowcolor[HTML]{E7E7E7} 
	ProjectController & /project\_overview $\rightarrow$ Zeigt alle Projekte an, für welche der WebUser die nötigen Berechtigungen hat \newline\newline
	/project/\{pID\}/show $\rightarrow$ Zeigt die Verträge des Projekts an, für welche der WebUser die nötigen Berechtigungen hat \\
	
	ContractController & /contract\_overview $\rightarrow$ Zeigt alle Verträge an, für welche der WebUser die nötigen Berechtigungen hat \newline\newline
	/project/\{pID\}/contract/\{cID\}/show $\rightarrow$ Zeigt die Leistungspositionen des Vertrags an, für welche der WebUser die nötigen Berechtigungen hat \\
	
	\rowcolor[HTML]{E7E7E7} 
	BillingItemController & /billing\_item\_overview $\rightarrow$ Zeigt alle Leistungspositionen an, für welche der WebUser die nötigen Berechtigungen hat \newline\newline
	/project/\{pID\}/contract/\{cID\}/billing\_item/\{bID\}/show $\rightarrow$ Zeigt Details zur Leistungsposition an, falls der WebUser die nötigen Berechtigungen hat \\
	
	DiagramController & /project\_diagram\_overview $\rightarrow$ Zeigt alle Diagramme zu Projekten an, für welche der WebUser die nötigen Berechtigungen hat \newline\newline
	/contract\_diagram\_overview $\rightarrow$ Zeigt alle Diagramme zu Verträgen an, für welche der WebUser die nötigen Berechtigungen hat
\end{longtable}

\clearpage

\section{App}

\begin{figure}[h]
	\includegraphics[width=\linewidth]{img/diagrams/Classdiagram-App.pdf}
	\caption{Klassendiagramm - App}
	\label{fig:klassendiagramm-a}
\end{figure}

\clearpage

\begin{table}[h]
	\centering
	\begin{tabularx}{\textwidth}{X X}
		\rowcolor[HTML]{C0C0C0} 
		\textbf{Klassenname} & \textbf{Aufgabe} \\
		Project & Bauplan eines Projektes mit den jeweiligen Attributen.\\
		\rowcolor[HTML]{E7E7E7} 
		Contract & Bauplan eines Vertrages mit den jeweiligen Attributen. \\
		BillingItem & Bauplan einer Leistungsposition mit den jeweiligen Attributen. \\
		\rowcolor[HTML]{E7E7E7} 
		ConstructionProgress & Bauplan einer Baufortschritts-Klasse mit den jeweiligen Attributen. \\
		LoginActivity & Verwaltung des Login-Bildschirms. \\
		\rowcolor[HTML]{E7E7E7} 
		ProjectActivity & Verwaltung des Projekt-Bildschirms. \\
		ContractActivity & Verwaltung des Vertags-Bildschirms. \\
		\rowcolor[HTML]{E7E7E7} 
		BillingItem & Verwaltung des Leistungspositions-Bildschirms. \\
		ConstructionProgressActivity & Verwaltung des Baufortschritts-Bildschirms. \\
		\rowcolor[HTML]{E7E7E7} 
		RequestHandler & Verwaltung der Netzwerkanfragen aller Klassen. \\
		ProjectDao & Datenzugriffsobjekt mit Anfragen zur Interaktion mit der Projektdatenbank. \\
		\rowcolor[HTML]{E7E7E7} 
		ProjectDatabase & Room-Datenbank für Projekte. \\
		ContractDao & Datenzugriffsobjekt mit Anfragen zur Interaktion mit der Vertragsdatenbank. \\
		\rowcolor[HTML]{E7E7E7} 
		ContractDatabase & Room-Datenbank für Verträge. \\
		BillingItemDao & Datenzugriffsobjekt mit Anfragen zur Interaktion mit der Leistungspositionsdatenbank. \\
		\rowcolor[HTML]{E7E7E7} 
		BillingItemDatabase & Room-Datenbank für Leistungspositionen. \\
		ConstructionProgressDao & Datenzugriffsobjekt mit Anfragen zur Interaktion mit der Baufortschritsdatenbank. \\
		\rowcolor[HTML]{E7E7E7} 
		ConstructionProgressDatabase & Room-Datenbank für den Baufortschritt. \\
		Status & Enumeration des Typs Status.
	\end{tabularx}
	\caption{Klassenbeschreibung - App}
	\label{table:klassenbeschreibung-a}
\end{table}
