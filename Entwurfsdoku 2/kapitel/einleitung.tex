\section{Dokumentaufbau}\label{sec:dokumentaufbau}
\begin{tcolorbox}
	Inhalt und Struktur des vorliegenden Dokuments skizzieren (Fließtext).
\end{tcolorbox}

\section{Zweckbestimmung}\label{sec:zweckbestimmung}
\begin{tcolorbox}
	Zweck des ganzen Systems beschreiben (Fließtext).
\end{tcolorbox}

\section{Entwicklungsumgebung}\label{sec:entwicklungsumgebung}
\begin{tcolorbox}
	Oftmals treten neue Entwickler einem Projekt bei oder ein Entwicklungs-Rechner muss ersetzt werden.
	Daher sollen hier nennenswerte und grundlegende Frameworks, Bibliotheken, Tools und Sprachen notiert werden.
	Tabelle X stellt eine beispielhafte Umsetzung dar.
	Eine Unterteilung in Komponenten ist sinnvoll.
\end{tcolorbox}

\begin{table}[h]
	\centering
	\begin{tabularx}{\textwidth}{l l X}
		\rowcolor[HTML]{C0C0C0} 
		\textbf{Software} & \textbf{Version} & \textbf{URL} \\
		Java Development Kit & 8u144 & \url{http://www.oracle.com/technetwork/java/javase/downloads/index.html} \\
		\rowcolor[HTML]{E7E7E7} 
		Software X & Version X & URL X \\
		Software X & Version X & URL X \\
		\rowcolor[HTML]{E7E7E7} 
		Software X & Version X & URL X \\
		Software X & Version X & URL X \\
		\rowcolor[HTML]{E7E7E7} 
		Software X & Version X & URL X \\
		Software X & Version X & URL X \\
	\end{tabularx}
	\caption{Enwicklungsumgebung}
	\label{table:entwicklungsumgebung}
\end{table}