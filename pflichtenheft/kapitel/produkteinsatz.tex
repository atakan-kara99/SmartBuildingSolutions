\section{Anwendungsgebiete}
Die Software soll die Verwaltung von Vertragsabläufen und Baufortschritten im Bauwesen vereinfachen.
Diese Prozesse digital durchzuführen, bietet diverse Vorteile.
So kann die Verwaltung etwa effizienter durchgeführt werden, da alles dafür Notwendige anschaulich an einem Ort angezeigt wird - es entsteht kein Papierchaos (Green through IT).\\
Darüber hinaus erhöht die Digitalisierung die Gesamttransparenz der Projekte.
In diesem Zuge können sowohl Auftragnehmer als auch Auftraggeber die Qualität der erbrachten Leistungen einfacher kontrollieren, wodurch Unzufriedenheiten leichter gemeldet und gelöst werden können.

\section{Zielgruppen}
Die Software wird von Auftraggebern, Auftragnehmern und deren Mitarbeitern genutzt.
Hierbei hat der Auftragnehmer oder ein ausgewählter Mitarbeiter die Rolle eines Organisationsadministrators (OrgAdmin) und weist anderen Mitarbeitern Rollen innerhalb der Organisation zu.
Außerdem gibt es einen Anwendungsadministrator, der beliebige Mitarbeiter wiederum zu Organisationsadministratoren ernennen kann (SysAdmin).\\
Sowohl Anwendungs- als auch Organisationsadministratoren sollten grundlegende technische Kenntnisse besitzen, um die Software sinnvoll verwenden zu können.
Die App und allgemeine Weboberfläche hingegen erfordern nicht diese Qualifikationen und kann von Mitarbeitern somit einfach bedient werden.
